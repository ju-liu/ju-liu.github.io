\documentclass[12pt]{article}
\pagestyle{empty}
\usepackage{amsmath,times,bm,hyperref}
\usepackage{amssymb}
\usepackage{listings}
\usepackage{xcolor}
\lstset { %
    language=C++,
    backgroundcolor=\color{black!5}, % set backgroundcolor
    basicstyle=\footnotesize,% basic font setting
}

%%%%%%%%%%%%%%%%%%%%%%%%%%%%%%%%%%%%%%%%%%%%%%%%%%
% Do not modify the dimensions of the page
\setlength{\topmargin}{0mm}
\setlength{\headheight}{0mm}
\setlength{\headsep}{0mm}
%% 25.4 -25.4 = 0
\setlength{\topmargin}{0mm}
%% 25.4 -25.4 = 0
\setlength{\oddsidemargin}{0mm}
%% 210 -25(left) -25(right) = 160
\setlength{\textwidth}{160mm}
%% 297 -25(top) -30(bottom) = 242
\setlength{\textheight}{242mm}
\setlength{\parindent}{0pt}
\setlength{\parskip}{12pt}
% Do not modify the dimensions of the page
%%%%%%%%%%%%%%%%%%%%%%%%%%%%%%%%%%%%%%%%%%%%%%%%%%

\begin{document}

\begin{center}
% TITLE: replace text with your abstract title WITHOUT full stop
\textbf{\Large
Homework 1
}\\
\normalsize Due March 16 2021

% AUTHOR/AFFILIATION: handled by authblk. 
% Use only one of the two following methods for author listing. Delete or comment out the other.
% Add/remove authors/affiliations as necessary, complete following the template without adding additional superscript/footnotes  

% 1- Authors have the same affiliation:
% ~~~~~~~~~~~~~~~~~~~~~~~~~~~~~~~~~~~~


%%%%% AFFILIATIONS %%%%%

\end{center}
\begin{enumerate}
\item We consider the \textbf{Bernoulli-Euler beam} theory posed on $\Omega = (0,1)$. The strong-form problem is stated as follows.

Given $f : \Omega \rightarrow \mathbb R$, constants $M$ and $Q$, constant Young's modulus $E$, and constant moment of inertia $I$, find $u : \overline{\Omega} \rightarrow \mathbb R$ such that
\begin{align*}
E I u_{,xxxx} &= f, & \mbox{ in } \Omega \quad & \mbox{(transverse equilibrium)} \\
u(1) & = 0, &  & \mbox{(zero transverse displacement)} \\
u_{,x}(1) &= 0, &  & \mbox{(zero slope)} \\
E I u_{,xx}(0) &= M, & & \mbox{(prescribed moment)} \\
E I u_{,xxx}(0) &= Q. & & \mbox{(prescribed shear)}
\end{align*}


Before considering a weak-form problem, let us first define the trial solution and test function spaces, which are given by
\begin{align*}
\mathcal S = \mathcal V = \Big\lbrace w : w \in H^2(\Omega), w(1) = w_{,x}(1) = 0 \Big\rbrace.
\end{align*}
Then the weak-form problem is stated as follows. 

Given $f : \Omega \rightarrow \mathbb R$, constants $M$ and $Q$, constant Young's modulus $E$, and constant moment of inertia $I$, find $u \in \mathcal S$ such that for all $w \in \mathcal V$,
\begin{align*}
a(w,u) = (w,f) - w_{,x}(0) M + w(0)Q,
\end{align*}
in which
\begin{align*}
a(w,u) := \int_{0}^{1} w_{,xx}E I u_{,xx} dx, \quad (w,f) := \int_0^1 wf dx.
\end{align*}

\begin{enumerate}
\item Derive the Euler-Lagrange equation for the weak-form problem.

\item What are the essential and the natural boundary conditions for the weak-form problem?

\item Show that the strong-form and weak-form problems are equivalent if the solution is sufficiently smooth (i.e., $C^4$).

\item The trial solution and test function spaces are both subsets of $H^2$. What does this imply for their smoothness (i.e. regularity) according to the Sobolev embedding theorem?

\end{enumerate}

\item The \textbf{Dirac delta function} $\delta$ can be useful in representing a point load. It is actually \textbf{not} a function in the classical sense. Instead, it is an operator acting on a continuous function. Let $w$ be a continuous function in $\Omega = (-1, 1)$, the action of $\delta$ on $w$ is defined as
\begin{align*}
\left( w, \delta \right) := w(0).
\end{align*}
Indeed, for continuous functions, the above definition is well-defined. The delta function belongs to $H^{-1}(\Omega)$.

\begin{enumerate}
\item Consider the Heaviside, or unit step, function:
\begin{align*}
H(x) = \begin{cases}
0 & \text{ if } -1 \leq x < 0, \\
1 & \text{ if } 0 \leq x \leq 1.
\end{cases}
\end{align*}
For any function $w \in \lbrace v : v \in C^0(\Omega), v(-1) = v(1) = 0 \rbrace$,
show that
\begin{align*}
\int_{-1}^{1} w H_{,x} dx = \int_{-1}^1 w \delta dx = w(0).
\end{align*}
This problem suggests that, although the Heaviside function is not differentiable at $x=0$, it has a \textbf{generalized} derivative that is defined under the integral and should be understood as an operator. The generalized derivative of $H$ is nothing but the Dirac delta function.

\item We may shift the delta function to a new location $y \in \Omega$ in the following way,
\begin{align*}
\delta_y(x) := \delta(x-y).
\end{align*}
Show that 
\begin{align*}
\left( w, \delta_y \right) = w(y).
\end{align*}
Of course we may similarily shift the Heaviside function as
\begin{align*}
H_y(x) = H(x-y) = \begin{cases}
0 & \text{ if } -1 \leq x < y, \\
1 & \text{ if } y \leq x \leq 1.
\end{cases}
\end{align*}
Use the above definition, show that $\delta_y$ is the generalized derivative of $H_y$.
\end{enumerate} 

\item (\textbf{Superconvergence of the finite element method in 1D}) We consider the \textbf{Green's function} problem corresponding to the strong-form problem we considered in our class. We assume here that the heat conductivity is homogeneous and takes the value 1, after a proper dimensional analysis.
\begin{enumerate}
\item Consider the following BVP,
\begin{align*}
-g_{,xx} &= \delta_y(x) = \delta(x-y), &\mbox{ for } 0 < x < 1, 0 < y < 1, \\
g(1) &= 0, & \\
g_{,x}(0) &= 0. & 
\end{align*}
Here we use $g$ to represent the solution, because the solution of this problem is called the Green's function. We may formally or symbolically solve this problem, if we acknowledge the relation $H_{y,x} = \delta_y$. Derive the analytic form of $g$.

\item If we recall that the test function space is defined as $\mathcal V = \lbrace w : w \in H^1, w(1) = 0 \rbrace$. Does $g$ belong to $\mathcal V$?

\item Let $a(\cdot, \cdot)$ be the bilinear form used in the corresponding weak-form problem. Show that $a(w,g) = w(y)$.

\item We consider that the space $\mathcal V^h$ is constructed by the piecewise linear finite element space. If we consider the domain $\Omega = (0,1)$ is partitioned into $n$ elements denoted by $[x_A, x_{A+1}]$ for $A=1, \cdots,  n$. We also require that $x_1 = 0$ and $x_n = 1$. The shape functions are defined as
\begin{align*}
N_A(x) = 
\begin{cases}
\frac{x-x_{A-1}}{h_{A-1}}, & x_{A-1} \leq x \leq x_A, \\
\frac{x_{A+1}-x}{h_A}, & x_A \leq x \leq x_{A+1}, \\
0, & \mbox{ otherwise. }
\end{cases}
\end{align*}
In the above $h_A =x_{A+1} -x_{A}$ is the length of the element. Then a typical member $w^h \in \mathcal V^h$ can be represented as $w_h = \sum_{A=1}^{n}c_A N_A$. Verify that $g \in \mathcal V^h$ if we choose $y = x_A$.
\item Let $u$ be the solution of the weak-form problem, and let $u^h$ be the finite element solution. Show that $a(w^h, u-u^h) = 0$ holds for any $w^h \in \mathcal V^h$.

\item We choose $y=x_A$ and $g$ is the corresponding Green's function. Show that
\begin{align*}
u(x_A) = u^h(x_A).
\end{align*}
This means for any one-dimensional problem (e.g. rod under axial compression, heat transfer, flow in porous media, etc.), finite element gives nodally exact solution. This is called the superconvergence of the finite element method. Notice that this result can hold hold for one-dimensional problems.

\item (optional) The superconvergence result we derived above \textbf{cannot} be generalized to two- or three-dimensional cases. Why?
\end{enumerate}

Hint: The problem 3 is in fact discussed in Section 1.10 of the book ``The Finite Element Method: Linear Static and Dynamic Finite Element Analysis".
\end{enumerate}

\end{document}

